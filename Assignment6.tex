\documentclass{beamer}
\usepackage{xspace}

 \usepackage{calc}                                             %%
    \usepackage{multirow}                                         %%
    \usepackage{hhline}                                           %%
    \usepackage{ifthen}
    \newcommand{\myvec}[1]{\ensuremath{\begin{pmatrix}#1\end{pmatrix}}}
\usetheme{CambridgeUS}

\title{Assignment 6}
\author{Saanvi Amrutha-AI21BTECH11026}
\date{\today}
\logo{\large \LaTeX{}}

\usepackage{amsmath}
\setbeamertemplate{caption}[numbered]{}
\providecommand{\pr}[1]{\ensuremath{\Pr\left(#1\right)}}
\providecommand{\cbrak}[1]{\ensuremath{\left\{#1\right\}}}

\begin{document}

\begin{frame}
    \titlepage 
\end{frame}

\logo{}

\begin{frame}{Outline}
    \tableofcontents
\end{frame}

\section{Question}
\begin{frame}{Question}
    \begin{block}{Papoulis Chapter 3 Example 3.13}
       An order of $10^4$ parts is received. The probability that a part is defective equals $0.1$. What is the probability that the total number of defective parts does not exceed $1100$?
    \end{block}
\end{frame}

\section{Solution}
\begin{frame}
\frametitle{Solution}
Given, 
\begin{align}
\pr{defective}&=p=0.1\\
\pr{non-defective}&=q=1-p\\
&=0.9\\
\end{align}
\end{frame}
\begin{frame}
 \begin{enumerate}
     \item Let the number of defective parts be $k$.Then,\\
     \begin{align}
         \pr{k_1\leq k\leq k_2}=\sum_{k=k_1}^{k_2}\myvec{n\\k}\brak{p^k}\brak{q^{n-k}}\\
     \end{align}
     Here,
     \begin{align}
     n&=10^{4}\\
     k1&=0\\
     k2&=1100
     \end{align} 
     \begin{align}
      \implies\pr{0\leq k\leq 1100}=\sum_{k=0}^{1100}\myvec{10^4\\k}\brak{0.1}^k\brak{0.9}^{n-k}  
     \end{align}
 \end{enumerate}   
\end{frame}
\end{document}